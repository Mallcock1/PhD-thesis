\documentclass[12pt]{../style-files/ociamthesis}
 
\usepackage{amssymb}
\usepackage{titlesec}
\usepackage{amsmath}
\usepackage{float}
\usepackage{graphicx}
\usepackage{caption}
\usepackage{subfig}
\usepackage{xcolor}
\usepackage[section]{placeins}
\usepackage{mathrsfs}
\usepackage{bm}
\usepackage{stmaryrd}
\usepackage{siunitx}
\usepackage{rotating}
\usepackage[utf8]{inputenc}
\usepackage[round]{natbib}

\usepackage{geometry}
 \geometry{
 a4paper,
 left=40mm,
 right=30mm,
 top=30mm,
 bottom=30mm
 }

\definecolor{theblue}{HTML}{0000CD}

% disable this package for printed version
\usepackage[colorlinks=true, linktocpage=true, allcolors=theblue]{hyperref}

\titleformat{\chapter}[display]
  {\bfseries\Large}
  {\filright\MakeUppercase{\chaptertitlename} \Large\thechapter}
  {1ex}
  {}
  [\vspace{1ex} \hrule \vspace{1pt} \hrule]

\newcommand{\adv}{    {\it Adv. Space Res.}} 
\newcommand{\annG}{   {\it Ann. Geophys.}} 
\newcommand{\aap}{    {\it Astron. Astrophys.}}
\newcommand{\aaps}{   {\it Astron. Astrophys. Suppl.}}
\newcommand{\aapr}{   {\it Astron. Astrophys. Rev.}}
\newcommand{\ag}{     {\it Ann. Geophys.}}
\newcommand{\aj}{     {\it Astron. J.}} 
\newcommand{\apj}{    {\it Astrophys. J.}}
\newcommand{\apjl}{   {\it Astrophys. J. Lett.}}
\newcommand{\apss}{   {\it Astrophys. Space Sci.}} 
\newcommand{\cjaa}{   {\it Chin. J. Astron. Astrophys.}} 
\newcommand{\gafd}{   {\it Geophys. Astrophys. Fluid Dyn.}}
\newcommand{\grl}{    {\it Geophys. Res. Lett.}}
\newcommand{\ijga}{   {\it Int. J. Geomagn. Aeron.}}
\newcommand{\jastp}{  {\it J. Atmos. Solar-Terr. Phys.}} 
\newcommand{\jgr}{    {\it J. Geophys. Res.}}
\newcommand{\mnras}{  {\it Mon. Not. Roy. Astron. Soc.}}
\newcommand{\na}{     {\it New Astronomy}}
\newcommand{\nat}{    {\it Nature}}
\newcommand{\pasp}{   {\it Pub. Astron. Soc. Pac.}}
\newcommand{\pasj}{   {\it Pub. Astron. Soc. Japan}}
\newcommand{\pre}{    {\it Phys. Rev. E}}
\newcommand{\solphys}{{\it Solar Phys.}}
\newcommand{\sovast}{ {\it Soviet  Astron.}} 
\newcommand{\ssr}{    {\it Space Sci. Rev.}}
\newcommand{\caa}{    {\it Chinese Astron. Astrohpys.}} 
\newcommand{\apjs}{   {\it Astrophys. J. Suppl.}}

\begin{document}

\baselineskip=18pt

\setcounter{secnumdepth}{3}
\setcounter{tocdepth}{3}

\setcounter{chapter}{6}


%------------------------------------------------------------------------------
\chapter{Future work}
\label{chap: future work}
%------------------------------------------------------------------------------

%\section{Promising directions}

\section{Creating a catalogue of observations of asymmetric MHD waves in the solar atmosphere}
The bulk of the present thesis is focussed on developing the theory of solar MHD waves. One promising direction would be to approach this concept from an observational point of view. A key first step in this direction is to catalogue the array of asymmetric waves. With a large enough sample, this could answer questions such
\begin{itemize}
	\item How asymmetric are solar waveguides?
	\item Do different types of solar structures exhibit different magnitudes of asymmetry?
\end{itemize}

In this thesis, we discussed several mechanisms through which waves in the solar atmosphere could appear asymmetric, for example, the wave could  be guided by an asymmetric waveguide, it could be a symmetric waveguide that has been asymmetrically perturbed, or it could be a localised wave rather than a collective wave. A large enough observational study, coupled with an understanding of the observational signatures of each of these mechanisms, could shed light on what mechanism is the most dominant in different solar structures.

Asymmetry of solar waves has not been addressed observationally largely due to the high spatial resolution needed to resolve the variation in wave power across a waveguide. The modern fleet of solar observational instrumentation is now able to do this, although the quality of image in the required scale is still poor. This will become less of a problem in the coming years as the next generation of Earth-based telescopes, where higher spatial resolution can be achieved,  are utilised.


\section{Realistic asymmetric waveguides}
One of the main flaws of the present work is the simplicity of the waveguide. Whilst this has allowed for increased mathematical tractability using a range of different mathematical approaches, it has to trade-off against the applicability of the model. Going forward, modelling more realistic asymmetric waveguides would lead to a better understanding of the asymmetric waves in the solar atmosphere and allow for the development of more accurate magneto-seismological techniques. Two more realistic asymmetric waveguides that would be value to study are:
\begin{itemize}
	\item \textit{An asymmetric slab with transitional regions}. Replacing the strict discontinuities at the boundaries of an asymmetric slab with a continuous monotonic function would open the problem up to phase mixing in the transitional regions. This well-studied dissipation mechanism  would presumably lead to differential heating across the waveguide, something that is yet to be studied but might explain observations of localised heating due to MHD wave dissipation in solar structures.
	\item \textit{A magnetic flux tube in an exponentially changing background}. Many of the waveguides in the solar atmosphere are more like cylinders than slabs, yet may still guide asymmetric waves, in the sense that the waves could have different amplitudes on two sides of the cylindrical cross-section. I am less optimistic about the mathematical tractability of this problem due to the loss of axisymmetry. Further, the background parameter gradient would apply differential pressure around the flux tube boundary. Therefore, for it to remain in equilibrium, the boundary of the tube must be non-circular and, presumably, a parameter gradient would be induced inside the tube. As you can see, merely deriving a mathematical description of the equilibrium would be quite some task.
\end{itemize}


%%------------------------------------------------------------------------------
%\section{Prioritisation in solar physics}
%\label{sec: prio}
%%------------------------------------------------------------------------------
%
%Progress in solar physics is driven by some combination of theory, which has been the focus of this thesis, numerical simulations, and observations. During the modern era of solar physics, the optimum distribution of resources spread between these three cornerstones has changed markedly. 
%
%There is clearly still important work to be done in each of these domains. However, to ensure the most successful progress, a thoughtful approach to prioritisation between them must be taken. 



\bibliographystyle{agsm}
\bibliography{../main/references}

\end{document}
