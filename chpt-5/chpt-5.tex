\documentclass[12pt]{../style-files/ociamthesis}
 
\usepackage{amssymb}
\usepackage{titlesec}
\usepackage{amsmath}
\usepackage{float}
\usepackage{graphicx}
\usepackage{caption}
\usepackage{subfig}
\usepackage{xcolor}
\usepackage[section]{placeins}
\usepackage{mathrsfs}
\usepackage{bm}
\usepackage{stmaryrd}
\usepackage{siunitx}
\usepackage{rotating}
\usepackage[utf8]{inputenc}
\usepackage[round]{natbib}
\usepackage{tikz}
\usetikzlibrary{fadings}
\usetikzlibrary{arrows,shapes, positioning}
\usepackage{booktabs}
\usepackage{multirow}
\usepackage{rotating}
\usepackage{tabularx}
\usepackage{makecell}

%%%%%%%%%%%%%%%%%%%%%%%%%%%%%%%%%%%%%%%%%%%%%%%%%%%%%%%%%%%%%%%%%%%%%%%
% the following is to alter tikz settings to improve springs figure.
\usetikzlibrary{decorations.pathmorphing,calc,patterns}
\makeatletter
\def\pgfdecorationspringstraightlinelength{0.5cm}
\def\pgfdecorationspringnumberofelement{8}
\def\pgfdecorationspringnaturallength{5cm}
\pgfkeys{%
	/pgf/decoration/.cd,
	spring straight line length/.code={%
		\pgfmathsetlengthmacro\pgfdecorationspringstraightlinelength{#1}},
	spring natural length/.code={%
		\pgfmathsetlengthmacro\pgfdecorationspringnaturallength{#1}},
	spring number of element/.store in=\pgfdecorationspringnumberofelement
}

\pgfdeclaredecoration{coil spring}{straight line}{%
	\state{straight line}[%
	persistent precomputation = {%
		% Compute the effective length of the spring (without the length
		% of the two straight lines): \pgfdecorationspringeffectivelength
		\pgfmathsetlengthmacro{\pgfdecorationspringeffectivelength}%
		{\pgfdecoratedpathlength-2*\pgfdecorationspringstraightlinelength}
		% Compute the effective length of one coil pattern:
		% \pgfdecorationspringeffectivelengthofonecoil
		\pgfmathsetlengthmacro{\pgfdecorationspringeffectivelengthofonecoil}%
		{\pgfdecorationspringeffectivelength/\pgfdecorationspringnumberofelement}
	},
	width = \pgfdecorationspringstraightlinelength,
	next state = draw spring]{%
		\pgfpathlineto{%
			\pgfqpoint{%
				\pgfdecorationspringstraightlinelength}{0pt}}
	}
	\state{draw spring}%
	[width=\pgfdecorationspringeffectivelengthofonecoil,
	repeat state=\pgfdecorationspringnumberofelement-1,next state=final]{%
		\pgfpathcurveto
		{\pgfpoint@onspringcoil{0    }{ 0.555}{1}}
		{\pgfpoint@onspringcoil{0.445}{ 1    }{2}}
		{\pgfpoint@onspringcoil{1    }{ 1    }{3}}
		\pgfpathcurveto
		{\pgfpoint@onspringcoil{1.555}{ 1    }{4}}
		{\pgfpoint@onspringcoil{2    }{ 0.555}{5}}
		{\pgfpoint@onspringcoil{2    }{ 0    }{6}}
		\pgfpathcurveto
		{\pgfpoint@onspringcoil{2    }{-0.555}{7}}
		{\pgfpoint@onspringcoil{1.555}{-1    }{8}}
		{\pgfpoint@onspringcoil{1    }{-1    }{9}}
		\pgfpathcurveto
		{\pgfpoint@onspringcoil{0.445}{-1    }{10}}
		{\pgfpoint@onspringcoil{0    }{-0.555}{11}}
		{\pgfpoint@onspringcoil{0    }{ 0    }{12}}
	}
	\state{final}{%
		\pgfpathlineto{\pgfpointdecoratedpathlast}
	}
}

\def\pgfpoint@onspringcoil#1#2#3{%
	\pgf@x=#1\pgfdecorationsegmentamplitude%
	\pgf@x=.5\pgf@x%
	\pgf@y=#2\pgfdecorationsegmentamplitude%
	\pgfmathparse{0.083333333333*\pgfdecorationspringeffectivelengthofonecoil}%
	\pgf@xa=\pgfmathresult pt
	\advance\pgf@x by#3\pgf@xa%
}

\makeatother

\tikzset{%
	Spring/.style = {%
		decoration = {%
			coil spring,
			spring straight line length = 0.2cm,
			% To be added
			spring natural length = #1,
			spring number of element = 4,
			amplitude=2mm},
		decorate,
		very thick},
	Spring/.default = {4cm}}
%
%%%%%%%%%%%%%%%%%%%%%%%%%%%%%%%%%%%%%%%%%%%%%%%%%%%%%%%%%%%%%%%%%%%%%%%%%%%%%%%%

\usepackage{geometry}
 \geometry{
 a4paper,
 left=40mm,
 right=30mm,
 top=30mm,
 bottom=30mm
 }

\definecolor{theblue}{HTML}{0000CD}

% disable this package for printed version
\usepackage[colorlinks=true, linktocpage=true, allcolors=theblue]{hyperref}

\titleformat{\chapter}[display]
  {\bfseries\Large}
  {\filright\MakeUppercase{\chaptertitlename} \Large\thechapter}
  {1ex}
  {}
  [\vspace{1ex} \hrule \vspace{1pt} \hrule]

\newcommand{\adv}{    {\it Adv. Space Res.}} 
\newcommand{\annG}{   {\it Ann. Geophys.}} 
\newcommand{\aap}{    {\it Astron. Astrophys.}}
\newcommand{\aaps}{   {\it Astron. Astrophys. Suppl.}}
\newcommand{\aapr}{   {\it Astron. Astrophys. Rev.}}
\newcommand{\ag}{     {\it Ann. Geophys.}}
\newcommand{\aj}{     {\it Astron. J.}} 
\newcommand{\apj}{    {\it Astrophys. J.}}
\newcommand{\apjl}{   {\it Astrophys. J. Lett.}}
\newcommand{\apss}{   {\it Astrophys. Space Sci.}} 
\newcommand{\cjaa}{   {\it Chin. J. Astron. Astrophys.}} 
\newcommand{\gafd}{   {\it Geophys. Astrophys. Fluid Dyn.}}
\newcommand{\grl}{    {\it Geophys. Res. Lett.}}
\newcommand{\ijga}{   {\it Int. J. Geomagn. Aeron.}}
\newcommand{\jastp}{  {\it J. Atmos. Solar-Terr. Phys.}} 
\newcommand{\jgr}{    {\it J. Geophys. Res.}}
\newcommand{\mnras}{  {\it Mon. Not. Roy. Astron. Soc.}}
\newcommand{\na}{     {\it New Astronomy}}
\newcommand{\nat}{    {\it Nature}}
\newcommand{\pasp}{   {\it Pub. Astron. Soc. Pac.}}
\newcommand{\pasj}{   {\it Pub. Astron. Soc. Japan}}
\newcommand{\pre}{    {\it Phys. Rev. E}}
\newcommand{\solphys}{{\it Solar Phys.}}
\newcommand{\sovast}{ {\it Soviet  Astron.}} 
\newcommand{\ssr}{    {\it Space Sci. Rev.}}
\newcommand{\caa}{    {\it Chinese Astron. Astrohpys.}} 
\newcommand{\apjs}{   {\it Astrophys. J. Suppl.}}
\newcommand{\lrsp}{{\it Living Rev. Solar Phys.}}

\newcommand{\bv}{\mathbf{v}}
\newcommand{\bB}{\mathbf{B}}

\newcommand{\figdirV}{../main/figures/chpt-5/} % where figures are stored


\begin{document}

\baselineskip=18pt

\setcounter{secnumdepth}{3}
\setcounter{tocdepth}{3}

\setcounter{chapter}{4}


% include tex file for chapter
\documentclass[12pt]{../style-files/ociamthesis}
 
\usepackage{amssymb}
\usepackage{titlesec}
\usepackage{amsmath}
\usepackage{float}
\usepackage{graphicx}
\usepackage{caption}
\usepackage{subfig}
\usepackage{xcolor}
\usepackage[section]{placeins}
\usepackage{mathrsfs}
\usepackage{bm}
\usepackage{stmaryrd}
\usepackage{siunitx}
\usepackage{rotating}
\usepackage[utf8]{inputenc}
\usepackage[round]{natbib}
\usepackage{tikz}
\usetikzlibrary{fadings}
\usetikzlibrary{arrows,shapes, positioning}
\usepackage{booktabs}
\usepackage{multirow}
\usepackage{rotating}
\usepackage{tabularx}
\usepackage{makecell}

%%%%%%%%%%%%%%%%%%%%%%%%%%%%%%%%%%%%%%%%%%%%%%%%%%%%%%%%%%%%%%%%%%%%%%%
% the following is to alter tikz settings to improve springs figure.
\usetikzlibrary{decorations.pathmorphing,calc,patterns}
\makeatletter
\def\pgfdecorationspringstraightlinelength{0.5cm}
\def\pgfdecorationspringnumberofelement{8}
\def\pgfdecorationspringnaturallength{5cm}
\pgfkeys{%
	/pgf/decoration/.cd,
	spring straight line length/.code={%
		\pgfmathsetlengthmacro\pgfdecorationspringstraightlinelength{#1}},
	spring natural length/.code={%
		\pgfmathsetlengthmacro\pgfdecorationspringnaturallength{#1}},
	spring number of element/.store in=\pgfdecorationspringnumberofelement
}

\pgfdeclaredecoration{coil spring}{straight line}{%
	\state{straight line}[%
	persistent precomputation = {%
		% Compute the effective length of the spring (without the length
		% of the two straight lines): \pgfdecorationspringeffectivelength
		\pgfmathsetlengthmacro{\pgfdecorationspringeffectivelength}%
		{\pgfdecoratedpathlength-2*\pgfdecorationspringstraightlinelength}
		% Compute the effective length of one coil pattern:
		% \pgfdecorationspringeffectivelengthofonecoil
		\pgfmathsetlengthmacro{\pgfdecorationspringeffectivelengthofonecoil}%
		{\pgfdecorationspringeffectivelength/\pgfdecorationspringnumberofelement}
	},
	width = \pgfdecorationspringstraightlinelength,
	next state = draw spring]{%
		\pgfpathlineto{%
			\pgfqpoint{%
				\pgfdecorationspringstraightlinelength}{0pt}}
	}
	\state{draw spring}%
	[width=\pgfdecorationspringeffectivelengthofonecoil,
	repeat state=\pgfdecorationspringnumberofelement-1,next state=final]{%
		\pgfpathcurveto
		{\pgfpoint@onspringcoil{0    }{ 0.555}{1}}
		{\pgfpoint@onspringcoil{0.445}{ 1    }{2}}
		{\pgfpoint@onspringcoil{1    }{ 1    }{3}}
		\pgfpathcurveto
		{\pgfpoint@onspringcoil{1.555}{ 1    }{4}}
		{\pgfpoint@onspringcoil{2    }{ 0.555}{5}}
		{\pgfpoint@onspringcoil{2    }{ 0    }{6}}
		\pgfpathcurveto
		{\pgfpoint@onspringcoil{2    }{-0.555}{7}}
		{\pgfpoint@onspringcoil{1.555}{-1    }{8}}
		{\pgfpoint@onspringcoil{1    }{-1    }{9}}
		\pgfpathcurveto
		{\pgfpoint@onspringcoil{0.445}{-1    }{10}}
		{\pgfpoint@onspringcoil{0    }{-0.555}{11}}
		{\pgfpoint@onspringcoil{0    }{ 0    }{12}}
	}
	\state{final}{%
		\pgfpathlineto{\pgfpointdecoratedpathlast}
	}
}

\def\pgfpoint@onspringcoil#1#2#3{%
	\pgf@x=#1\pgfdecorationsegmentamplitude%
	\pgf@x=.5\pgf@x%
	\pgf@y=#2\pgfdecorationsegmentamplitude%
	\pgfmathparse{0.083333333333*\pgfdecorationspringeffectivelengthofonecoil}%
	\pgf@xa=\pgfmathresult pt
	\advance\pgf@x by#3\pgf@xa%
}

\makeatother

\tikzset{%
	Spring/.style = {%
		decoration = {%
			coil spring,
			spring straight line length = 0.2cm,
			% To be added
			spring natural length = #1,
			spring number of element = 4,
			amplitude=2mm},
		decorate,
		very thick},
	Spring/.default = {4cm}}
%
%%%%%%%%%%%%%%%%%%%%%%%%%%%%%%%%%%%%%%%%%%%%%%%%%%%%%%%%%%%%%%%%%%%%%%%%%%%%%%%%

\usepackage{geometry}
 \geometry{
 a4paper,
 left=40mm,
 right=30mm,
 top=30mm,
 bottom=30mm
 }

\definecolor{theblue}{HTML}{0000CD}

% disable this package for printed version
\usepackage[colorlinks=true, linktocpage=true, allcolors=theblue]{hyperref}

\titleformat{\chapter}[display]
  {\bfseries\Large}
  {\filright\MakeUppercase{\chaptertitlename} \Large\thechapter}
  {1ex}
  {}
  [\vspace{1ex} \hrule \vspace{1pt} \hrule]

\newcommand{\adv}{    {\it Adv. Space Res.}} 
\newcommand{\annG}{   {\it Ann. Geophys.}} 
\newcommand{\aap}{    {\it Astron. Astrophys.}}
\newcommand{\aaps}{   {\it Astron. Astrophys. Suppl.}}
\newcommand{\aapr}{   {\it Astron. Astrophys. Rev.}}
\newcommand{\ag}{     {\it Ann. Geophys.}}
\newcommand{\aj}{     {\it Astron. J.}} 
\newcommand{\apj}{    {\it Astrophys. J.}}
\newcommand{\apjl}{   {\it Astrophys. J. Lett.}}
\newcommand{\apss}{   {\it Astrophys. Space Sci.}} 
\newcommand{\cjaa}{   {\it Chin. J. Astron. Astrophys.}} 
\newcommand{\gafd}{   {\it Geophys. Astrophys. Fluid Dyn.}}
\newcommand{\grl}{    {\it Geophys. Res. Lett.}}
\newcommand{\ijga}{   {\it Int. J. Geomagn. Aeron.}}
\newcommand{\jastp}{  {\it J. Atmos. Solar-Terr. Phys.}} 
\newcommand{\jgr}{    {\it J. Geophys. Res.}}
\newcommand{\mnras}{  {\it Mon. Not. Roy. Astron. Soc.}}
\newcommand{\na}{     {\it New Astronomy}}
\newcommand{\nat}{    {\it Nature}}
\newcommand{\pasp}{   {\it Pub. Astron. Soc. Pac.}}
\newcommand{\pasj}{   {\it Pub. Astron. Soc. Japan}}
\newcommand{\pre}{    {\it Phys. Rev. E}}
\newcommand{\solphys}{{\it Solar Phys.}}
\newcommand{\sovast}{ {\it Soviet  Astron.}} 
\newcommand{\ssr}{    {\it Space Sci. Rev.}}
\newcommand{\caa}{    {\it Chinese Astron. Astrohpys.}} 
\newcommand{\apjs}{   {\it Astrophys. J. Suppl.}}
\newcommand{\lrsp}{{\it Living Rev. Solar Phys.}}

\newcommand{\bv}{\mathbf{v}}
\newcommand{\bB}{\mathbf{B}}

\newcommand{\figdirV}{../main/figures/chpt-5/} % where figures are stored


\begin{document}

\baselineskip=18pt

\setcounter{secnumdepth}{3}
\setcounter{tocdepth}{3}

\setcounter{chapter}{4}


% include tex file for chapter
\documentclass[12pt]{../style-files/ociamthesis}
 
\usepackage{amssymb}
\usepackage{titlesec}
\usepackage{amsmath}
\usepackage{float}
\usepackage{graphicx}
\usepackage{caption}
\usepackage{subfig}
\usepackage{xcolor}
\usepackage[section]{placeins}
\usepackage{mathrsfs}
\usepackage{bm}
\usepackage{stmaryrd}
\usepackage{siunitx}
\usepackage{rotating}
\usepackage[utf8]{inputenc}
\usepackage[round]{natbib}
\usepackage{tikz}
\usetikzlibrary{fadings}
\usetikzlibrary{arrows,shapes, positioning}
\usepackage{booktabs}
\usepackage{multirow}
\usepackage{rotating}
\usepackage{tabularx}
\usepackage{makecell}

%%%%%%%%%%%%%%%%%%%%%%%%%%%%%%%%%%%%%%%%%%%%%%%%%%%%%%%%%%%%%%%%%%%%%%%
% the following is to alter tikz settings to improve springs figure.
\usetikzlibrary{decorations.pathmorphing,calc,patterns}
\makeatletter
\def\pgfdecorationspringstraightlinelength{0.5cm}
\def\pgfdecorationspringnumberofelement{8}
\def\pgfdecorationspringnaturallength{5cm}
\pgfkeys{%
	/pgf/decoration/.cd,
	spring straight line length/.code={%
		\pgfmathsetlengthmacro\pgfdecorationspringstraightlinelength{#1}},
	spring natural length/.code={%
		\pgfmathsetlengthmacro\pgfdecorationspringnaturallength{#1}},
	spring number of element/.store in=\pgfdecorationspringnumberofelement
}

\pgfdeclaredecoration{coil spring}{straight line}{%
	\state{straight line}[%
	persistent precomputation = {%
		% Compute the effective length of the spring (without the length
		% of the two straight lines): \pgfdecorationspringeffectivelength
		\pgfmathsetlengthmacro{\pgfdecorationspringeffectivelength}%
		{\pgfdecoratedpathlength-2*\pgfdecorationspringstraightlinelength}
		% Compute the effective length of one coil pattern:
		% \pgfdecorationspringeffectivelengthofonecoil
		\pgfmathsetlengthmacro{\pgfdecorationspringeffectivelengthofonecoil}%
		{\pgfdecorationspringeffectivelength/\pgfdecorationspringnumberofelement}
	},
	width = \pgfdecorationspringstraightlinelength,
	next state = draw spring]{%
		\pgfpathlineto{%
			\pgfqpoint{%
				\pgfdecorationspringstraightlinelength}{0pt}}
	}
	\state{draw spring}%
	[width=\pgfdecorationspringeffectivelengthofonecoil,
	repeat state=\pgfdecorationspringnumberofelement-1,next state=final]{%
		\pgfpathcurveto
		{\pgfpoint@onspringcoil{0    }{ 0.555}{1}}
		{\pgfpoint@onspringcoil{0.445}{ 1    }{2}}
		{\pgfpoint@onspringcoil{1    }{ 1    }{3}}
		\pgfpathcurveto
		{\pgfpoint@onspringcoil{1.555}{ 1    }{4}}
		{\pgfpoint@onspringcoil{2    }{ 0.555}{5}}
		{\pgfpoint@onspringcoil{2    }{ 0    }{6}}
		\pgfpathcurveto
		{\pgfpoint@onspringcoil{2    }{-0.555}{7}}
		{\pgfpoint@onspringcoil{1.555}{-1    }{8}}
		{\pgfpoint@onspringcoil{1    }{-1    }{9}}
		\pgfpathcurveto
		{\pgfpoint@onspringcoil{0.445}{-1    }{10}}
		{\pgfpoint@onspringcoil{0    }{-0.555}{11}}
		{\pgfpoint@onspringcoil{0    }{ 0    }{12}}
	}
	\state{final}{%
		\pgfpathlineto{\pgfpointdecoratedpathlast}
	}
}

\def\pgfpoint@onspringcoil#1#2#3{%
	\pgf@x=#1\pgfdecorationsegmentamplitude%
	\pgf@x=.5\pgf@x%
	\pgf@y=#2\pgfdecorationsegmentamplitude%
	\pgfmathparse{0.083333333333*\pgfdecorationspringeffectivelengthofonecoil}%
	\pgf@xa=\pgfmathresult pt
	\advance\pgf@x by#3\pgf@xa%
}

\makeatother

\tikzset{%
	Spring/.style = {%
		decoration = {%
			coil spring,
			spring straight line length = 0.2cm,
			% To be added
			spring natural length = #1,
			spring number of element = 4,
			amplitude=2mm},
		decorate,
		very thick},
	Spring/.default = {4cm}}
%
%%%%%%%%%%%%%%%%%%%%%%%%%%%%%%%%%%%%%%%%%%%%%%%%%%%%%%%%%%%%%%%%%%%%%%%%%%%%%%%%

\usepackage{geometry}
 \geometry{
 a4paper,
 left=40mm,
 right=30mm,
 top=30mm,
 bottom=30mm
 }

\definecolor{theblue}{HTML}{0000CD}

% disable this package for printed version
\usepackage[colorlinks=true, linktocpage=true, allcolors=theblue]{hyperref}

\titleformat{\chapter}[display]
  {\bfseries\Large}
  {\filright\MakeUppercase{\chaptertitlename} \Large\thechapter}
  {1ex}
  {}
  [\vspace{1ex} \hrule \vspace{1pt} \hrule]

\newcommand{\adv}{    {\it Adv. Space Res.}} 
\newcommand{\annG}{   {\it Ann. Geophys.}} 
\newcommand{\aap}{    {\it Astron. Astrophys.}}
\newcommand{\aaps}{   {\it Astron. Astrophys. Suppl.}}
\newcommand{\aapr}{   {\it Astron. Astrophys. Rev.}}
\newcommand{\ag}{     {\it Ann. Geophys.}}
\newcommand{\aj}{     {\it Astron. J.}} 
\newcommand{\apj}{    {\it Astrophys. J.}}
\newcommand{\apjl}{   {\it Astrophys. J. Lett.}}
\newcommand{\apss}{   {\it Astrophys. Space Sci.}} 
\newcommand{\cjaa}{   {\it Chin. J. Astron. Astrophys.}} 
\newcommand{\gafd}{   {\it Geophys. Astrophys. Fluid Dyn.}}
\newcommand{\grl}{    {\it Geophys. Res. Lett.}}
\newcommand{\ijga}{   {\it Int. J. Geomagn. Aeron.}}
\newcommand{\jastp}{  {\it J. Atmos. Solar-Terr. Phys.}} 
\newcommand{\jgr}{    {\it J. Geophys. Res.}}
\newcommand{\mnras}{  {\it Mon. Not. Roy. Astron. Soc.}}
\newcommand{\na}{     {\it New Astronomy}}
\newcommand{\nat}{    {\it Nature}}
\newcommand{\pasp}{   {\it Pub. Astron. Soc. Pac.}}
\newcommand{\pasj}{   {\it Pub. Astron. Soc. Japan}}
\newcommand{\pre}{    {\it Phys. Rev. E}}
\newcommand{\solphys}{{\it Solar Phys.}}
\newcommand{\sovast}{ {\it Soviet  Astron.}} 
\newcommand{\ssr}{    {\it Space Sci. Rev.}}
\newcommand{\caa}{    {\it Chinese Astron. Astrohpys.}} 
\newcommand{\apjs}{   {\it Astrophys. J. Suppl.}}
\newcommand{\lrsp}{{\it Living Rev. Solar Phys.}}

\newcommand{\bv}{\mathbf{v}}
\newcommand{\bB}{\mathbf{B}}

\newcommand{\figdirV}{../main/figures/chpt-5/} % where figures are stored


\begin{document}

\baselineskip=18pt

\setcounter{secnumdepth}{3}
\setcounter{tocdepth}{3}

\setcounter{chapter}{4}


% include tex file for chapter
\documentclass[12pt]{../style-files/ociamthesis}
 
\usepackage{amssymb}
\usepackage{titlesec}
\usepackage{amsmath}
\usepackage{float}
\usepackage{graphicx}
\usepackage{caption}
\usepackage{subfig}
\usepackage{xcolor}
\usepackage[section]{placeins}
\usepackage{mathrsfs}
\usepackage{bm}
\usepackage{stmaryrd}
\usepackage{siunitx}
\usepackage{rotating}
\usepackage[utf8]{inputenc}
\usepackage[round]{natbib}
\usepackage{tikz}
\usetikzlibrary{fadings}
\usetikzlibrary{arrows,shapes, positioning}
\usepackage{booktabs}
\usepackage{multirow}
\usepackage{rotating}
\usepackage{tabularx}
\usepackage{makecell}

%%%%%%%%%%%%%%%%%%%%%%%%%%%%%%%%%%%%%%%%%%%%%%%%%%%%%%%%%%%%%%%%%%%%%%%
% the following is to alter tikz settings to improve springs figure.
\usetikzlibrary{decorations.pathmorphing,calc,patterns}
\makeatletter
\def\pgfdecorationspringstraightlinelength{0.5cm}
\def\pgfdecorationspringnumberofelement{8}
\def\pgfdecorationspringnaturallength{5cm}
\pgfkeys{%
	/pgf/decoration/.cd,
	spring straight line length/.code={%
		\pgfmathsetlengthmacro\pgfdecorationspringstraightlinelength{#1}},
	spring natural length/.code={%
		\pgfmathsetlengthmacro\pgfdecorationspringnaturallength{#1}},
	spring number of element/.store in=\pgfdecorationspringnumberofelement
}

\pgfdeclaredecoration{coil spring}{straight line}{%
	\state{straight line}[%
	persistent precomputation = {%
		% Compute the effective length of the spring (without the length
		% of the two straight lines): \pgfdecorationspringeffectivelength
		\pgfmathsetlengthmacro{\pgfdecorationspringeffectivelength}%
		{\pgfdecoratedpathlength-2*\pgfdecorationspringstraightlinelength}
		% Compute the effective length of one coil pattern:
		% \pgfdecorationspringeffectivelengthofonecoil
		\pgfmathsetlengthmacro{\pgfdecorationspringeffectivelengthofonecoil}%
		{\pgfdecorationspringeffectivelength/\pgfdecorationspringnumberofelement}
	},
	width = \pgfdecorationspringstraightlinelength,
	next state = draw spring]{%
		\pgfpathlineto{%
			\pgfqpoint{%
				\pgfdecorationspringstraightlinelength}{0pt}}
	}
	\state{draw spring}%
	[width=\pgfdecorationspringeffectivelengthofonecoil,
	repeat state=\pgfdecorationspringnumberofelement-1,next state=final]{%
		\pgfpathcurveto
		{\pgfpoint@onspringcoil{0    }{ 0.555}{1}}
		{\pgfpoint@onspringcoil{0.445}{ 1    }{2}}
		{\pgfpoint@onspringcoil{1    }{ 1    }{3}}
		\pgfpathcurveto
		{\pgfpoint@onspringcoil{1.555}{ 1    }{4}}
		{\pgfpoint@onspringcoil{2    }{ 0.555}{5}}
		{\pgfpoint@onspringcoil{2    }{ 0    }{6}}
		\pgfpathcurveto
		{\pgfpoint@onspringcoil{2    }{-0.555}{7}}
		{\pgfpoint@onspringcoil{1.555}{-1    }{8}}
		{\pgfpoint@onspringcoil{1    }{-1    }{9}}
		\pgfpathcurveto
		{\pgfpoint@onspringcoil{0.445}{-1    }{10}}
		{\pgfpoint@onspringcoil{0    }{-0.555}{11}}
		{\pgfpoint@onspringcoil{0    }{ 0    }{12}}
	}
	\state{final}{%
		\pgfpathlineto{\pgfpointdecoratedpathlast}
	}
}

\def\pgfpoint@onspringcoil#1#2#3{%
	\pgf@x=#1\pgfdecorationsegmentamplitude%
	\pgf@x=.5\pgf@x%
	\pgf@y=#2\pgfdecorationsegmentamplitude%
	\pgfmathparse{0.083333333333*\pgfdecorationspringeffectivelengthofonecoil}%
	\pgf@xa=\pgfmathresult pt
	\advance\pgf@x by#3\pgf@xa%
}

\makeatother

\tikzset{%
	Spring/.style = {%
		decoration = {%
			coil spring,
			spring straight line length = 0.2cm,
			% To be added
			spring natural length = #1,
			spring number of element = 4,
			amplitude=2mm},
		decorate,
		very thick},
	Spring/.default = {4cm}}
%
%%%%%%%%%%%%%%%%%%%%%%%%%%%%%%%%%%%%%%%%%%%%%%%%%%%%%%%%%%%%%%%%%%%%%%%%%%%%%%%%

\usepackage{geometry}
 \geometry{
 a4paper,
 left=40mm,
 right=30mm,
 top=30mm,
 bottom=30mm
 }

\definecolor{theblue}{HTML}{0000CD}

% disable this package for printed version
\usepackage[colorlinks=true, linktocpage=true, allcolors=theblue]{hyperref}

\titleformat{\chapter}[display]
  {\bfseries\Large}
  {\filright\MakeUppercase{\chaptertitlename} \Large\thechapter}
  {1ex}
  {}
  [\vspace{1ex} \hrule \vspace{1pt} \hrule]

\newcommand{\adv}{    {\it Adv. Space Res.}} 
\newcommand{\annG}{   {\it Ann. Geophys.}} 
\newcommand{\aap}{    {\it Astron. Astrophys.}}
\newcommand{\aaps}{   {\it Astron. Astrophys. Suppl.}}
\newcommand{\aapr}{   {\it Astron. Astrophys. Rev.}}
\newcommand{\ag}{     {\it Ann. Geophys.}}
\newcommand{\aj}{     {\it Astron. J.}} 
\newcommand{\apj}{    {\it Astrophys. J.}}
\newcommand{\apjl}{   {\it Astrophys. J. Lett.}}
\newcommand{\apss}{   {\it Astrophys. Space Sci.}} 
\newcommand{\cjaa}{   {\it Chin. J. Astron. Astrophys.}} 
\newcommand{\gafd}{   {\it Geophys. Astrophys. Fluid Dyn.}}
\newcommand{\grl}{    {\it Geophys. Res. Lett.}}
\newcommand{\ijga}{   {\it Int. J. Geomagn. Aeron.}}
\newcommand{\jastp}{  {\it J. Atmos. Solar-Terr. Phys.}} 
\newcommand{\jgr}{    {\it J. Geophys. Res.}}
\newcommand{\mnras}{  {\it Mon. Not. Roy. Astron. Soc.}}
\newcommand{\na}{     {\it New Astronomy}}
\newcommand{\nat}{    {\it Nature}}
\newcommand{\pasp}{   {\it Pub. Astron. Soc. Pac.}}
\newcommand{\pasj}{   {\it Pub. Astron. Soc. Japan}}
\newcommand{\pre}{    {\it Phys. Rev. E}}
\newcommand{\solphys}{{\it Solar Phys.}}
\newcommand{\sovast}{ {\it Soviet  Astron.}} 
\newcommand{\ssr}{    {\it Space Sci. Rev.}}
\newcommand{\caa}{    {\it Chinese Astron. Astrohpys.}} 
\newcommand{\apjs}{   {\it Astrophys. J. Suppl.}}
\newcommand{\lrsp}{{\it Living Rev. Solar Phys.}}

\newcommand{\bv}{\mathbf{v}}
\newcommand{\bB}{\mathbf{B}}

\newcommand{\figdirV}{../main/figures/chpt-5/} % where figures are stored


\begin{document}

\baselineskip=18pt

\setcounter{secnumdepth}{3}
\setcounter{tocdepth}{3}

\setcounter{chapter}{4}


% include tex file for chapter
\input{../main/5-chpt/chpt-5.tex}


\bibliographystyle{agsm}
\bibliography{../main/references}

\end{document}



\bibliographystyle{agsm}
\bibliography{../main/references}

\end{document}



\bibliographystyle{agsm}
\bibliography{../main/references}

\end{document}



\bibliographystyle{agsm}
\bibliography{../main/references}

\end{document}
