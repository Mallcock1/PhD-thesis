\documentclass[12pt]{../style-files/ociamthesis}
 
\usepackage{amssymb}
\usepackage{titlesec}
\usepackage{amsmath}
\usepackage{float}
\usepackage{graphicx}
\usepackage{caption}
\usepackage{subfig}
\usepackage{xcolor}
\usepackage[section]{placeins}
\usepackage{mathrsfs}
\usepackage{bm}
\usepackage{stmaryrd}
\usepackage{siunitx}
\usepackage{rotating}
\usepackage[utf8]{inputenc}
\usepackage[round]{natbib}

\usepackage{geometry}
 \geometry{
 a4paper,
 left=40mm,
 right=30mm,
 top=30mm,
 bottom=30mm
 }

\definecolor{theblue}{HTML}{0000CD}

% disable this package for printed version
\usepackage[colorlinks=true, linktocpage=true, allcolors=theblue]{hyperref}

\titleformat{\chapter}[display]
  {\bfseries\Large}
  {\filright\MakeUppercase{\chaptertitlename} \Large\thechapter}
  {1ex}
  {}
  [\vspace{1ex} \hrule \vspace{1pt} \hrule]

\newcommand{\adv}{    {\it Adv. Space Res.}} 
\newcommand{\annG}{   {\it Ann. Geophys.}} 
\newcommand{\aap}{    {\it Astron. Astrophys.}}
\newcommand{\aaps}{   {\it Astron. Astrophys. Suppl.}}
\newcommand{\aapr}{   {\it Astron. Astrophys. Rev.}}
\newcommand{\ag}{     {\it Ann. Geophys.}}
\newcommand{\aj}{     {\it Astron. J.}} 
\newcommand{\apj}{    {\it Astrophys. J.}}
\newcommand{\apjl}{   {\it Astrophys. J. Lett.}}
\newcommand{\apss}{   {\it Astrophys. Space Sci.}} 
\newcommand{\cjaa}{   {\it Chin. J. Astron. Astrophys.}} 
\newcommand{\gafd}{   {\it Geophys. Astrophys. Fluid Dyn.}}
\newcommand{\grl}{    {\it Geophys. Res. Lett.}}
\newcommand{\ijga}{   {\it Int. J. Geomagn. Aeron.}}
\newcommand{\jastp}{  {\it J. Atmos. Solar-Terr. Phys.}} 
\newcommand{\jgr}{    {\it J. Geophys. Res.}}
\newcommand{\mnras}{  {\it Mon. Not. Roy. Astron. Soc.}}
\newcommand{\na}{     {\it New Astronomy}}
\newcommand{\nat}{    {\it Nature}}
\newcommand{\pasp}{   {\it Pub. Astron. Soc. Pac.}}
\newcommand{\pasj}{   {\it Pub. Astron. Soc. Japan}}
\newcommand{\pre}{    {\it Phys. Rev. E}}
\newcommand{\solphys}{{\it Solar Phys.}}
\newcommand{\sovast}{ {\it Soviet  Astron.}} 
\newcommand{\ssr}{    {\it Space Sci. Rev.}}
\newcommand{\caa}{    {\it Chinese Astron. Astrohpys.}} 
\newcommand{\apjs}{   {\it Astrophys. J. Suppl.}}

\begin{document}

\baselineskip=18pt

\setcounter{secnumdepth}{3}
\setcounter{tocdepth}{3}

\setcounter{chapter}{5}


%------------------------------------------------------------------------------
\chapter{Conclusions}
\label{chap: conclusion}
%------------------------------------------------------------------------------

The complex magnetic field structure of the solar atmosphere allows for the equilibrium suspension of regions of different plasma parameters that can guide MHD waves. The inhomogeneity is such that we cannot expect these waveguides to always be close to symmetrical. Breaking the symmetry of solar waveguide models increases the mathematical difficulty but provides valuable insights into these asymmetric solar waveguides. Given that this is the first exploration of asymmetry in solar waveguide models, we focussed on the most simple such model: the asymmetric slab.

Studying the asymmetric slab as an eigenvalue problem (EVP), the dispersion relation has solutions, which are the waveguide's eigenfrequencies, have mixed properties of the traditional (symmetric) sausage and kink modes. distinguishing features of the traditional sausage and kink modes are that the sausage mode perturbs the cross-sectional width and leaves the axis unperturbed, whereas the kink mode leaves the cross-sectional width unperturbed and perturbs the axis. In contrast, all of the eigenmodes of the asymmetric slab perturb both the axis and the cross-sectional width. However, there is still a sense in which we can define two categories of eigenmodes, namely, the phase relationship of the waveguide boundaries. Asymmetric eigenmodes are described as quasi-sausage (quasi-kink) if the oscillations of the waveguide boundaries are in anti-phase (phase). This suggests that the phase relationship of the waveguide boundaries is the more fundamental characteristic on which to describe MHD eigenmodes.  The mixed nature of the asymmetric eigenmodes is expressed by the fact that the dispersion relation does not decouple into separate equations for sausage and kink eigenfrequencies. This makes the dispersion relation for the asymmetric slab mathematically distinct from the dispersion relation for a symmetric slab \citep{rob81b,edw_etal82}.

The mixed properties of asymmetric eigenmodes could possibly lead to incorrect mode identification of waves in the solar atmosphere. In particular, since both the quasi-sausage and quasi-kink modes perturb the cross-sectional width and the waveguide axis, these modes could be practically indistinguishable from nonlinear symmetric modes or a superposition of linear symmetric eigenmodes. Therefore, identification must come from the phase relationship of the boundary oscillations rather than either the cross-sectional width or the axial perturbations.

A second way in which asymmetric modes could be misidentified is through the existence of quasi-symmetric eigenmodes. These are eigenmodes of an asymmetric waveguide that appear to be symmetric, in the sense that the amplitudes on each boundary are equal. This occurs when the sum of the magnetic and pressure gradient restoring forces is equal on both sides of an asymmetric waveguide. We derived necessary and sufficient conditions for this phenomenon to occur. The implication of this is that merely observing a symmetric wave in a solar waveguide is insufficient to deduce that the background parameters are symmetric.

The main difference in the dispersion diagram of the asymmetric eigenmodes in comparison to the symmetric eigenmodes is the presence of a cut-off frequency. Collective oscillations with frequency above the cut-off frequency in a sufficiently thin slab are not trapped by the waveguide. Instead, these oscillations leak energy laterally into the external plasma regions. Due to the asymmetry of the waveguide, the leakage occurs asymmetrically in the sense that energy is leaked at a different rate on each side. The asymmetry can be so stark that the wave can be completely trapped on one side whilst leaking out of the other.

Asymmetric wave leakage is more intuitively described by ray theory, which is a mathematical approach to wave-related problems that describes a wave as having only a speed and a direction for each point in time. By defining a phase-ray, the dispersion relation for the asymmetric slab is derived using a different approach to that of the eigenvalue problem. In this derivation, the ray is assumed to undergo total internal reflection when incident on the waveguide boundaries. Relaxing this requirement allows for some portion of the wave energy to be transmitted into the external plasma, leading to attenuation of the collective wave. The simplicity of ray theory in dispersion relation derivation and its intuitive explanation for phenomena such as leaky modes shows that the potential for this approach is perhaps underutilised in MHD theory.

The temporal evolution of a series of initially perturbed MHD waveguides was investigated. Initially perturbed waveguides are known to evolve through a series of three phases: the \textit{initial phase}, the period before collective modes are excited; the \textit{impulsive phase}, where leaky modes can dominate; and the \textit{stationary phase}, where trapped modes dominate for an indefinite time period. Firstly, we studied the initial value problem of an incompressible tangential interface. This relatively simple problem was first studied nearly 40 years ago by \cite{rae_etal81}. The key result from our solution to this problem is to correct a mistake that was made early in the original paper. We showed that the tangential interface which is initially perturbed with constant vorticity evolves through both surface and body modes, as opposed to just body modes. Since this problem was studied for an incompressible plasma, there is no wave leakage and any incompressible initial condition excites trapped modes instantaneously, so only the stationary phase exists.

Next, we solved the initial value problem for an incompressible asymmetric slab. Again, only the stationary phase exists because only trapped eigenmodes are excited, of which the time-dependent solution is a linear summation.

Next, we solved the initial value problem for a cold symmetric slab. The analysis resulted in an asymptotic solution that is valid for large values of time. The solution is made up of three kinds of terms corresponding to the three phases of evolution. We showed that the impulsive phase is much shorted in duration than for a similar initial condition in a cold magnetic flux tube. Of course, the precise nature of the three phases is highly dependent on the initial condition. Generalising this result to an asymmetric slab, we showed that for a sufficiently thin slab, the trapped principle kink mode becomes leaky. This means that for a sufficiently thin cold asymmetric slab, the impulsive phase is non-existent because all the excited collective modes are leaky. In this case, all the energy from the initial disturbance will eventually we transferred laterally into the background plasma.

The major application of this theory of asymmetric solar waveguides is in magneto-seismology. We developed two new techniques that use the eigenmode asymmetry as a proxy for the background magnetic field strength, which is difficult to measure using traditional methods. The techniques are known as the \textit{Amplitude Ratio Method}, which uses the ratio of the boundary amplitudes as a proxy for asymmetry, and the \textit{Minimum Perturbation Shift Method}, which uses the shift of the position of minimum perturbation as a proxy for asymmetry. As a proof of concept, we applied the Amplitude Ratio Method to a series of 5 chromospheric fibrils observed by the ROSA instrument on the Dunn Solar Telescope in 2012. The estimated Alfv\'{e}n speeds range from 30.5 and 91.7~kms$^{-1}$. These values fit in the ball-park of previous estimates using different techniques.


\bibliographystyle{agsm}
\bibliography{../main/references} 

\end{document}
