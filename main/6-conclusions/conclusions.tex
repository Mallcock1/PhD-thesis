%------------------------------------------------------------------------------
\chapter{Conclusions}
\label{chap: conclusion}
%------------------------------------------------------------------------------

The magnetic field of the solar atmosphere can support plasma structures in equilibrium. Stable perturbations of these structures may propagate as MHD waves. Many previous mathematical models of these waveguides utilised either reflectional symmetry or axisymmetry for mathematical simplicity, yet this assumption is not valid for a wide range of solar structures. Breaking the symmetry of solar waveguide models increases the mathematical difficulty but provides valuable insights into these asymmetric solar waveguides. Given that this thesis is the first exploration of asymmetry in solar waveguide models, we focussed on the most simple asymmetric MHD model: the asymmetric slab.

Firstly, studying the asymmetric slab as an eigenvalue problem (EVP), the dispersion relation has solutions which are the waveguide's eigenfrequencies, have mixed properties of the traditional (symmetric) sausage and kink modes \citep{all_etal17}. Distinguishing features of the traditional sausage and kink modes are that the sausage mode perturbs the waveguide's cross-sectional width and leaves the waveguide's axis unperturbed, whereas the kink mode leaves the cross-sectional width unperturbed and perturbs the axis. In contrast, all of the eigenmodes of the asymmetric slab perturb both the axis and the cross-sectional width. However, we can define two categories of asymmetric eigenmodes using the phase relationship of the waveguide boundaries. Asymmetric eigenmodes are described as quasi-sausage (quasi-kink) if the oscillations of the waveguide boundaries are in anti-phase (phase). This suggests that the phase relationship of the waveguide boundaries is a fundamental characteristic on which to describe MHD eigenmodes, rather than the presence of cross-sectional width or axial perturbation.  The mixed nature of the asymmetric eigenmodes is expressed mathematically by the fact that the dispersion relation does not decouple into separate equations for sausage and kink eigenfrequencies. This makes the dispersion relation for the asymmetric slab mathematically distinct from the dispersion relation for a symmetric slab \citep{rob81b,edw_etal82}.

We identify a concern that the mixed properties of asymmetric eigenmodes could lead to the incorrect identification of MHD modes in the solar atmosphere. In particular, since both the quasi-sausage and quasi-kink modes perturb the cross-sectional width and the waveguide axis, these modes would have similar observational features to nonlinear symmetric modes or a superposition of linear symmetric eigenmodes. Therefore, identification must include the phase relationship of the boundary oscillations rather than either the cross-sectional width or the axial perturbations.

A second way in which asymmetric modes could be misidentified is through the existence of quasi-symmetric eigenmodes \citep{zsa_etal18}. We describe eigenmodes of an asymmetric waveguide as quasi-symmetric when they appear to be symmetric, in the sense that the amplitudes on each waveguide boundary are equal. In the simplest case where the only restoring forces are the magnetic force and the pressure gradient force, this occurs when the sum of the magnetic and pressure gradient restoring forces is equal on both sides of an asymmetric waveguide. We derived necessary and sufficient conditions for this phenomenon to occur. The key implication of this is that merely observing a symmetric wave in a solar waveguide is insufficient to deduce without ambiguity that the background parameters are symmetric.

The main difference in the dispersion diagram of the asymmetric eigenmodes in comparison to the symmetric eigenmodes is the presence of a cut-off frequency. Collective oscillations with frequency above the cut-off frequency in a sufficiently thin slab are not trapped by the waveguide. Instead, these oscillations leak energy laterally into the external plasma regions. Due to the asymmetry of the waveguide, the leakage occurs asymmetrically in the sense that energy is leaked at a different rate on each side. The asymmetry can be so stark that part of the wave is be completely trapped on one side of the waveguide whilst leaking out of the other.

Asymmetric wave leakage can be described more intuitively by \textit{ray theory}. Ray theory is a mathematical description of waves as having only a speed and a direction for each point in time. By defining a phase-ray, the dispersion relation for the asymmetric slab is derived using a different approach to that of the eigenvalue problem. In this derivation, the ray is assumed to undergo total internal reflection when incident on the waveguide boundaries. Relaxing this requirement allows for some portion of the wave energy to be transmitted into the external plasma, leading to attenuation of the collective wave. The simplicity of ray theory in dispersion relation derivation and its intuitive explanation for phenomena such as leaky modes shows that the potential for this approach is perhaps underutilised in MHD theory.

The temporal evolution of a series of initially perturbed MHD waveguides was investigated. Initially perturbed waveguides that are not subject to any damping mechanism are known to evolve through a series of three phases: the \textit{initial phase}, the phase before collective modes are excited; the \textit{impulsive phase}, where leaky modes can dominate; and the \textit{stationary phase}, where trapped modes dominate for an indefinite time period (see, for example, \cite{rud_etal06b}). In this thesis, we studied the initial value problem of an incompressible tangential interface. This relatively simple problem was first studied nearly 40 years ago by \cite{rae_etal81}. The key result from our solution to this problem is to correct a mistake that was made early in the original paper. We showed that the tangential interface which is initially perturbed with constant vorticity drives both surface and body modes, rather than just body modes as claimed by \cite{rae_etal81}. Since this problem was studied for an incompressible plasma, there is no wave leakage and any incompressible initial condition excites trapped modes instantaneously, so only the stationary phase exists in this case.

Next, we solved the initial value problem for an incompressible asymmetric slab. Again, only the stationary phase exists because only trapped eigenmodes are excited, of which the time-dependent solution is a linear summation.

Finally, we solved the initial value problem for a cold symmetric slab. The analysis resulted in an asymptotic solution that is valid for large values of time. The solution is made up of three groups of terms corresponding to the three phases of evolution, as expected. We showed that the impulsive phase is much shorter in duration than for a similar initial condition in a cold magnetic flux tube. Of course, the precise nature of the three phases is highly dependent on the initial conditions. Generalising this result to an asymmetric slab, we showed that for a sufficiently thin slab, the trapped principle kink mode becomes leaky. This means that for a sufficiently thin cold asymmetric slab, the impulsive phase is non-existent because all the excited collective modes are leaky. In this case, all the energy from the initial disturbance will eventually we transferred laterally into the background plasma, rather than continuing to propagate along the waveguide.

The major application of the theory of asymmetric MHD waveguides developed in this thesis is in solar magneto-seismology \citep{all_etal18a}. We developed two new techniques that use the eigenmode asymmetry as a proxy for the background magnetic field strength, which is difficult to measure using traditional methods. The Amplitude Ratio Method uses the ratio of the boundary amplitudes as a proxy for asymmetry and the Minimum Perturbation Shift Method uses the shift of the position of minimum perturbation as a proxy for asymmetry. We applied the Amplitude Ratio Method to a series of 5 chromospheric fibrils observed by the ROSA instrument on the Dunn Solar Telescope in 2012 \citep{all_etal19}. The estimated Alfv\'{e}n speeds range from 30.5 and 91.7~kms$^{-1}$. These values fit in the ball-park of previous estimates using different techniques.