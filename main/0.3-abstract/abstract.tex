\begin{abstract}

Over 50 years of solar magnetohydrodynamic wave theory has focussed on waveguides in symmetric plasma environments. Yet the Sun’s inhomogeneous atmosphere supports waveguides held in asymmetric equilibrium. In this thesis, we break this symmetry by studying a slab waveguide model embedded in an asymmetric external plasma with three approaches:
\begin{itemize}
	\item Eigenvalue problem: We derive the dispersion relation and show that asymmetric eigenmodes have mixed properties of the traditional sausage and kink modes.
	\item Ray theory: We demonstrate how a ray theoretic approach can be used to derive this dispersion relation, giving an intuitive description of asymmetric leaky modes.
	\item Initial value problem: An initial perturbation of an asymmetric slab evolves, in general, through a series of three phases: the \textit{initial phase}, the period before collective modes are excited; the \textit{impulsive phase}, where leaky modes can dominate; and the \textit{stationary phase}, where trapped modes dominate for an indefinite time period. We show that, in general, the impulsive phase for a slab is significantly shorter than for a magnetic flux tube. We then show that an asymmetric slab of cold plasma does not have a stationary phase because the principle kink mode in an asymmetric slab is leaky.
\end{itemize}
Next, we derive two magneto-seismology techniques to estimate the magnetic field strength in asymmetric solar waveguides. We apply this novel technique to a series of solar chromospheric fibrils as a proof of concept with estimated values of the Alfv\'{e}n speed that agree with estimates using traditional techniques.


\end{abstract}