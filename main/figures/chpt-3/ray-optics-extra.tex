\documentclass[12pt]{article}

\usepackage{hyperref}
\usepackage{amssymb}
\usepackage{amsthm, amsmath}
\usepackage{graphicx}
\usepackage[round]{natbib}
\usepackage{tikz}
\usepackage{natbib}

\newcommand{\adv}{    {\it Adv. Space Res.}} 
\newcommand{\annG}{   {\it Ann. Geophys.}} 
\newcommand{\aap}{    {\it Astron. Astrophys.}}
\newcommand{\aaps}{   {\it Astron. Astrophys. Suppl.}}
\newcommand{\aapr}{   {\it Astron. Astrophys. Rev.}}
\newcommand{\ag}{     {\it Ann. Geophys.}}
\newcommand{\aj}{     {\it Astron. J.}} 
\newcommand{\apj}{    {\it Astrophys. J.}}
\newcommand{\apjl}{   {\it Astrophys. J. Lett.}}
\newcommand{\apss}{   {\it Astrophys. Space Sci.}} 
\newcommand{\cjaa}{   {\it Chin. J. Astron. Astrophys.}} 
\newcommand{\gafd}{   {\it Geophys. Astrophys. Fluid Dyn.}}
\newcommand{\grl}{    {\it Geophys. Res. Lett.}}
\newcommand{\ijga}{   {\it Int. J. Geomagn. Aeron.}}
\newcommand{\jastp}{  {\it J. Atmos. Solar-Terr. Phys.}} 
\newcommand{\jgr}{    {\it J. Geophys. Res.}}
\newcommand{\mnras}{  {\it Mon. Not. Roy. Astron. Soc.}}
\newcommand{\na}{     {\it New Astronomy}}
\newcommand{\nat}{    {\it Nature}}
\newcommand{\pasp}{   {\it Pub. Astron. Soc. Pac.}}
\newcommand{\pasj}{   {\it Pub. Astron. Soc. Japan}}
\newcommand{\pre}{    {\it Phys. Rev. E}}
\newcommand{\solphys}{{\it Solar Phys.}}
\newcommand{\sovast}{ {\it Soviet  Astron.}} 
\newcommand{\ssr}{    {\it Space Sci. Rev.}}
\newcommand{\caa}{    {\it Chinese Astron. Astrohpys.}} 
\newcommand{\apjs}{   {\it Astrophys. J. Suppl.}}
\newcommand{\lrsp}{{\it Living Rev. Solar Phys.}}
\newcommand{\planss}{{\it Planetary and Space Sci.}}



\newtheorem{theorem}{Theorem}

\title{Alternative derivation of MHD slab dispersion relation using ray optics - EXTRA}
\author{Matthew Allcock}

\begin{document}
	\maketitle

	\textcolor{red}{Disambiguate: Direction of propagation, wave-vector, wave-front normal, ellipses of group and phase, group and phase angles, slowness vector.
		\begin{itemize}
			\item $\mathbf{k}$ points in the direction of the phase velocity, $\mathbf{v_{ph}} = \omega/k\mathbf{\hat{k}}$, i.e. normal to surfaces of constant phase, known as wavefronts. AKA wave-normal.
			\item "Direction of propagation" often refers to the direction of a wave's energy flow, i.e. the direction of the group velocity $\mathbf{v_g} = \partial \omega / \partial \mathbf{k} = (\partial \omega / \partial k_x, \partial \omega / \partial k_y, \partial \omega / \partial k_z)$. This it the component of the velocity of energy propagation (Poynting vector?) in the wave-normal direction.
			\item The direction of energy propagation is given by the Poynting vector $\mathbf{S} = (\mathbf{E} \times \mathbf{B})/\mu_0$.
			\item There is another speed called the ray velocity, which is defined by $v_r = v_{ph}/\cos{\alpha}$ where $\alpha$ is the angle between $v_{ph}$ and $v_g$.
			\item A ray is often defined as a (infinitely long) line collinear with the wave-vector $\mathbf{k}$. This is only the case for isotropic media. For anisotropic media, the rays can be non-perpendicular to wavefronts (e.g. \url{http://sepwww.stanford.edu/sep/prof/iei/dspr/paper_html/node22.html}). Instead, we should \textit{define} a ray as the curve between two points that satisfied Fermat's principle of least time. This involved calculus of variations to find for all but homogeneous media.
			\item The slowness vector is the reciprocal of the phase velocity, i.e. $\mathbf{w} = 1/v_{ph} \mathbf{\hat{k}}$. Commonly used in Earth seismology.
			\item the different direction of the group and phase speeds motivate definitions of group and phase angles, $\nu$ and $\theta$, that denote the angles between the the group velocity the phase velocity, and the direction tangential to the interface in the plane of incidence, respectively. Then by geometry, $\cos(\theta - \nu) = |v_{ph} / v_g|$ and $\cos{\theta_1}/v_{ph1} = \cos{\theta_2}/v_{ph_2}$, i.e. Snell's law still holds for phase angles (see e.g. Nunez 2018).
		\end{itemize}
	}
	
	\section{Low beta plasma}
	
	Consider an asymmetric slab MHD waveguide. This is an inviscid plasma made up of three regions labelled 1, 0, 2, separated by interfaces at $x = \pm x_0$ with magnetic field in the $z$-direction (see \citep{all_etal17}). In the first instance, assume the plasma beta is small. This degenerates the slow MHD mode and reduces the fast MHD mode to a compressional Alfv\'{e}n mode, which propagates isotropically with phase speed $\omega / k = v_A$, where $v_A$ is the Alfv\'{e}n speed. In particular, rays propagate in straight lines, so that ray optics theory can be implemented in it's standard form. 
	
	Consider a small amplitude ray propagating as an MHD fast mode at an angle $\theta_i$ to the interface between plasma 0 and 2. Low beta MHD waves have no velocity component parallel with the magnetic field, so the velocity perturbation associated with this wave can be written $\mathbf{v_i} = (v_{ix}, 0, 0)$. In general, when this ray hits the interface, it will be partially reflected and partially transmitted. Each of these waves can be decomposed into a linear superposition of plane waves, whose transverse components have the form
	\begin{align}
	v_{ix} &= \hat{v}_{ix} e^{i(\mathbf{k_i}\cdot \mathbf{x} + \omega t)}, \\
	v_{rx} &= -\hat{v}_{rx} e^{i(\mathbf{k_r}\cdot \mathbf{x} - \omega t)}, \\
	v_{tx} &= \hat{v}_{tx} e^{i(\mathbf{k_t}\cdot \mathbf{x} - \omega t)},
	\end{align}
	where subscripts $i, r, t$ refer to \textit{incident}, \textit{reflected}, and \textit{transmitted} wave. The factor of $-1 = e^{i\pi}$ in the reflected wave amplitude is due to the $\pi$ change in phase when the incident wave is reflected. Let the angles than the incident, reflected, and transmitted rays make with the interface be $\theta_i$, $\theta_r$, and $\theta_t$, respectively.
	
	As is standard procedure for free surfaces between fluids, the normal velocity and total pressure perturbation must both be continuous at the interface. The total pressure perturbation for a low-beta plasma is purely the magnetic pressure, since this dominates the plasma pressure. The linearised magnetic pressure is $p_m = B_0b_{iz}/\mu_0$, where $B_0$ and $b_{iz}$ are the equilibrium and $z$-component of the magnetic field in the slab region and $\mu_0$ is the magnetic permeability. Using the $z$-component of the induction equation,
	\begin{equation}
	\hat{b}_{iz} = -i\frac{B_0}{\omega} (\hat{v}'_{ix} + ik_{ix}\hat{v}_x),
	\end{equation}
	where $b_{iz} = \hat{b}_{iz} e^{i(\mathbf{k_i}\cdot \mathbf{x} + \omega t)}$. The incident wave is effectively a homogeneous plasma wave, so it's amplitude is uniform, \textit{i.e.} $\hat{v}'_{ix} = 0$, therefore
	\begin{equation}
	\hat{b}_{iz} = \frac{B_0}{\omega} k_{ix}\hat{v}_{ix}.
	\end{equation}
	Hence, the total pressure associated with the incident wave can be expressed as
	\begin{equation}
	p_{Ti} = \frac{\rho_0v_{A0}^2}{\omega}k_{ix}v_{ix}.
	\end{equation}
	
	The law of reflection tells us that $\theta_r = \theta_i$. Continuity of tangential wavevector components tells us that
	\begin{equation}
	k_i\cos{\theta_i} = k_r\cos{\theta_r} = k_t\cos{\theta_t}. \label{tang comp}
	\end{equation}
	Therefore, $k_r = k_i$. In particular, $k_{rx} = -k_{ix}$ and $k_{rz} = k_{iz}$.
	
	Continuity of normal velocity at $x = 0$ gives
	\begin{equation}
	\hat{v}_{ix}e^{i(k_{iz}z - \omega t)} - \hat{v}_{rx}e^{i(k_{rz}z - \omega t)} = \hat{v}_{tx}e^{i(k_{tz}z - \omega t)}.
	\end{equation}
	Using Equation~\eqref{tang comp}, this reduces to
	\begin{equation}
	\hat{v}_{ix} - \hat{v}_{rx} = \hat{v}_{tx}. \label{cont vel}
	\end{equation}
	
	Continuity of total pressure perturbation at $x = 0$ gives
	\begin{equation}
	\frac{\rho_0v_{A0}^2}{\omega}(k_{ix}\hat{v}_{ix} - k_{rx}\hat{v}_{rx}) = \frac{\rho_2v_{A2}^2}{\omega}k_{tx}\hat{v}_{tx}.
	\end{equation}
	By balance of equilibrium total pressure, $\rho_0v_{A0}^2 = \rho_2v_{A2}^2$, and using the fact that $k_{rx} = -k_{ix}$, we reach
	\begin{equation}
	k_{ix}(\hat{v}_{ix} + \hat{v}_{rx}) = k_{tx}\hat{v}_{tx}. \label{cont pressure}
	\end{equation}
	
	Equations~\eqref{cont vel} and~\eqref{cont pressure} can be solved simultaneously to find
	\begin{align}
	\hat{v}_{ix} &= \frac{1}{2}\hat{v}_{tx}\left(\frac{k_{tx}}{k_{ix}} + 1\right) \\
	\hat{v}_{rx} &= \frac{1}{2}\hat{v}_{tx}\left(\frac{k_{tx}}{k_{ix}} - 1\right).
	\end{align}
	Therefore, the reflection coefficient  is
	\begin{equation}
	r := \frac{\hat{v}_{rx}}{\hat{v}_{ix}} = \frac{k_{tx} - k_{ix}}{k_{tx} + k_{ix}}.
	\end{equation}
	
	From Equation~\eqref{tang comp}, Snell's Law follows, that is
	\begin{equation}
	\frac{\cos{\theta_i}}{v_{A0}} = \frac{\cos{\theta_t}}{v_{A2}}.
	\end{equation}
	Total internal reflection occurs for $\theta_i < \theta_c$, where $\cos{\theta_c} = k_t / k_i$. In this case,
	\begin{align}
	k_{tx}^2 &= k_t^2 \sin^2{\theta_t} \\
	&= k_t^2(1 - \cos^2{\theta_t}) \\
	&= k_t^2 - k_i^2\cos^2{\theta_i} \\
	&< k_t^2 - k_i^2\cos^2{\theta_c} \\
	&= k_t^2 - k_i^2 \left(\frac{k_t^2}{k_i^2}\right) \\
	&= 0.
	\end{align}
	Therefore, $k_{tx}$ is imaginary. Define $k_2$ by $k_{tx} = -i k_2$, with $k_2 \in \mathcal{R}$, and let $k_0 := k_{ix} = \sqrt{\omega^2/v_{A0}^2 - k_z^2}$, so that we can write
	\begin{equation}
	r = -\frac{k_0 + ik_2}{k_0 - ik_2},
	\end{equation}
	which has complex argument
	\begin{equation}
	\phi_2 = -2 \arctan\left(\frac{k_2}{k_0}\right).
	\end{equation}
	This is the phase shift that the incident ray undergoes after total internal reflection on the interface between plasma 0 and 2. Similarly, the phase shift that an incident ray undergoes after total internal reflection on the interface between plasma 0 and 1 is
	\begin{equation}
	\phi_1 = -2 \arctan\left(\frac{k_1}{k_0}\right).
	\end{equation}
	
	Using the same position labels as Figure~1.2.3 from \cite{mar74} {Theory of Dielectric Optical Waveguides}. We can determine, using basic geometry, that
	\begin{equation}
	s_1 := AB = 2x_0(\cos^2{\theta_i} - \sin^2{\theta_i})/\sin{\theta_i}
	\end{equation}
	and
	\begin{equation}
	s_2 := CD = 2x_0/\sin{\theta_i}.
	\end{equation}
	Consider a ray incident on interface 2 with total internal reflection, then interface with interface 1 with total internal reflection. The condition that the wave must be self-consistent\footnote{This is also known as the \textit{transverse resonance condition} in relation to optical waveguides (\textit{e.g.} R.R.A.Syms and J.R.Cozens Optical Guided Waves and Devices).}, which is equivalent to the doubly reflected wave being in phase with the wave had it not made any reflections, leads to
	\begin{equation}
	n_0 s_2 k_i  + \phi_2 + \phi_1 = n_0 s_1 k_i + 2N\pi, \label{self-consistency}
	\end{equation}
	where $n_0$ is the refractive index for compressional Alfv\'{e}n modes in the slab and $N$ is an integer. $n_0s_1$ is the optical path length for the path $AB$. By using the internal Alfv\'{e}n speed to normalise the refractive index, we can set $n_0 = 1$. Using basic trigonometry, $1 - (\cos^2{\theta_i} - \sin^2{\theta_i}) = 2\sin^2{\theta_i}$, therefore, Equation~\eqref{self-consistency} becomes
	\begin{equation}
	\arctan\left(\frac{k_2}{k_0}\right) + \arctan\left(\frac{k_1}{k_0}\right) = 2x_0k_i \sin{\theta_i} - N\pi.
	\end{equation}
	Applying $\tan$ to this equation, using the identity $\tan(\arctan{a} + \arctan{b}) = (a + b) / (1 - ab)$, and noticing that $k_i\sin{\theta_i} = k_0$, yields
	\begin{equation}
	\tan(2k_0x_0) = \frac{k_0 (k_1 + k_2)}{k_0^2 - k_1k_2}.
	\end{equation}
	By letting $k_0 = -im_0$, we reach
	\begin{equation}
	m_0 (k_1 + k_2) + (m_0^2 + k_1k_2)\tanh(2m_0x_0) = 0. \label{DR low beta}
	\end{equation}
	
	By observing that
	\begin{equation}
	m_0 = ik_0 = \sqrt{k_z^2 - \frac{\omega^2}{v_{A0}^2}}
	\end{equation}
	and
	\begin{equation}
	k_{1,2} = \sqrt{k_z^2 - \frac{\omega^2}{v_{A1,2}^2}},
	\end{equation}
	it can be seen that Equation~\eqref{DR low beta} is equivalent to the dispersion relation for low-beta MHD modes guided by an asymmetric magnetic slab waveguide.
	
	
	\section{Incompressible plasma}
	In incompressible plasma, the fast mode propagates instantaneously so no longer transfers energy and the slow mode propagates at a speed which depends on the angle between the propagation direction and the magnetic field, $\theta$, like $c_\text{ph} = v_A\cos{\theta}$. Therefore, this wave propagates anisotropically. Hence, we need to verify that the law of reflection still holds and derive a Snell's law for the interaction of this anisotropic wave with a planar interface.
	
	\subsection{Derivation of alternative Snell's law}
	Referring to the diagram of \url{https://en.wikipedia.org/wiki/Snells_law} (note that their $\theta$ is $\pi/2 -$ out $\theta$), let $T$ be the time required for an incompressible slow MHD ray to travel between points $Q$ and $P$. Then, denoting the propagation speeds in the incident and transmitted regions as $v_1$ and $v_2$,
	\begin{align}
	T &= \frac{\sqrt{x^2 + a^2}}{v_1} + \frac{\sqrt{b^2 + (l - x)^2}}{v_2} \\
	&= \frac{\sqrt{x^2 + a^2}}{v_{A1}\cos{\theta_1}} + \frac{\sqrt{b^2 + (l - x)^2}}{v_{A2}\cos{\theta_2}}
	&= \frac{\sqrt{x^2 + a^2}}{v_{A1}\cos{\theta_1}} + \frac{\sqrt{b^2 + (l - x)^2}}{v_{A2}\cos{\theta_2}}.
	\end{align}
	The incident and transmission angles obey
	\begin{equation}
	\cos{\theta_1} = \frac{x}{\sqrt{x^2 + a^2}}, \quad \cos{\theta_2} = \frac{l - x}{\sqrt{b^2 + (l - x)^2}},
	\end{equation}
	therefore,
	\begin{equation}
	T = \frac{x^2 + a^2}{v_{A1}x} + \frac{b^2 + (l - x)^2}{v_{A2}(l - x)}.
	\end{equation}
	Therefore,
	\begin{align}
	\frac{dT}{dx} &= \frac{1}{v_{A1}}\left( 1 - \frac{a^2}{x^2} \right) - \frac{1}{v_{A2}}\left( 1 - \frac{b^2}{(l - x)^2} \right) \\
	&= \frac{1}{v_{A1}}\left( 1 - \tan^2{\theta_1} \right) - \frac{1}{v_{A2}}\left( 1 - \tan^2{\theta_2} \right).
	\end{align}
	Setting this equal to 0, so as to minimise $T$, gives us Snell's law for incompressible MHD waves, namely
	\begin{equation}
	\frac{1}{v_{A1}}\left( 1 - \tan^2{\theta_1} \right) = \frac{1}{v_{A2}}\left( 1 - \tan^2{\theta_2} \right).
	\end{equation}
	
	The law of total reflection can be derived by considering a reflected ray rather than a transmitted ray. The speed of the reflected ray is the same as the speed of the incident ray, $v_{A1}$. Let the propagation angle of the reflected ray be $\theta_r$, then using a very similar derivation to that of Snell's law,
	\begin{equation}
	\frac{1}{v_{A1}}\left( 1 - \tan^2{\theta_1} \right) = \frac{1}{v_{A1}}\left( 1 - \tan^2{\theta_r} \right).
	\end{equation}
	Therefore, $1 - \tan^2{\theta_1} = 1 - \tan^2{\theta_r}$, which can only be satisfied if $\theta_r = \theta_1$. Hence, the law of reflection holds.
	
	We know that incompressible fast MHD modes 
	
	\begin{itemize}
		\item $\mathbf{k} \cdot \mathbf{v} = 0$.
		\item No pressure or density perturbation.
		\item $\mathbf{B_0} \cdot \mathbf{b} = 0$, i.e. magnetic field is perturbed only in direction perpendicular to the equilibrium magnetic field.
	\end{itemize}



\bibliographystyle{agsm}
\bibliography{bibliography}  

\end{document}
