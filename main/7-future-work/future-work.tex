%------------------------------------------------------------------------------
\chapter{Future work}
\label{chap: future work}
%------------------------------------------------------------------------------

%\section{Promising directions}

\section{Compiling a solar catalogue of observations of asymmetric MHD waves}
The bulk of this thesis is focussed on developing the theory of solar MHD waves. One promising direction would be to approach this concept from an observational point of view. A key first step in this direction is to catalogue the array of asymmetric wave observations. With a large enough sample, this could answer questions such as
\begin{itemize}
	\item To what extent are solar MHD waves asymmetric?
	\item Do different types of solar structures exhibit different degrees of asymmetry?
	\item Is the asymmetry due to asymmetry of the waveguide, the initial perturbation or driver, or something else?
\end{itemize}

In this thesis, we discussed several mechanisms through which MHD waves in the solar atmosphere could appear asymmetric, for example, the wave could  be guided by an asymmetric waveguide, it could be a symmetric waveguide that has been asymmetrically perturbed, or it could be a localised wave rather than a collective wave. A large enough observational study, coupled with an understanding of the observational signatures of each of these mechanisms, would shed light on which mechanism is the most dominant in different solar structures.

Asymmetry of solar MHD waves has not been addressed widely from an observational point of view due to the high spatial resolution needed to resolve the variation in wave power across a waveguide. The modern fleet of solar observational instrumentation (for example, the Swedish Solar Telescope and the Daniel K. Inouye Solar Telescope) is now able to accomplish this, although the quality of image in the required scale is still poor. This will become less of a problem in the coming years as the next generation of Earth-based telescopes with improved spatial resolution are utilised.


\section{Realistic asymmetric waveguides}
The main drawback of the present work is the simplicity of the asymmetric waveguide model. Whilst this approach has allowed for increased mathematical tractability using a range of different mathematical techniques, it has to trade-off against the applicability of the waveguide model. Going forward, modelling more realistic asymmetric waveguides would lead to a better understanding of the asymmetric waves in the solar atmosphere and allow for the development of more accurate magneto-seismological techniques. Two more realistic asymmetric waveguides that would be valuable to study are:
\begin{itemize}
	\item \textit{An asymmetric slab with transitional regions}. Replacing the strict discontinuities imposed at the boundaries of an asymmetric slab with a continuous monotonic function would introduce phase-mixing and resonant absorption in the transitional regions. These otherwise well-studied dissipation mechanisms would presumably lead to differential heating across the waveguide. Differential waveguide heating is yet to be studied but could explain observations of localised heating due to MHD wave dissipation in solar structures.
	\item \textit{A magnetic flux tube in a non-uniform background}. Many of the waveguides in the solar atmosphere have a closer resemblance to cylindrical models, rather than slab models. Cylindrical waveguides may still guide asymmetric waves, in the sense that the waves could have different amplitudes on two sides of the cylindrical cross-section. A cylindrical waveguide embedded in a non-uniform background plasma could provide an accurate model of this. However, the background parameter gradient would apply differential pressure around the flux tube boundary. Therefore, for the flux tube to remain in equilibrium, the boundary of the tube must be non-circular and, presumably, a parameter gradient would be induced inside the tube. Merely deriving a mathematical description of the equilibrium would be quite some task, as one can see.
\end{itemize}


%%------------------------------------------------------------------------------
%\section{Prioritisation in solar physics}
%\label{sec: prio}
%%------------------------------------------------------------------------------
%
%Progress in solar physics is driven by some combination of theory, which has been the focus of this thesis, numerical simulations, and observations. During the modern era of solar physics, the optimum distribution of resources spread between these three cornerstones has changed markedly. 
%
%There is clearly still important work to be done in each of these domains. However, to ensure the most successful progress, a thoughtful approach to prioritisation between them must be taken. 